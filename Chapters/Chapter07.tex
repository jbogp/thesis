%************************************************
\chapter{Conclusions and future work}\label{ch:conclusions}
%************************************************

\section{Conclusions}

	\subsection{Summary}
	

	\section{Discussion}
		\subsection{Validity of the model's independence hypothesis}
		In this model we assume that, conditional upon the allocation of a cell to a cluster, the gene expression levels can be described by independent Bernoulli distributions. This is a reasonable assumption in the context of the 86 genes chosen by Tomer et al. \cite{Tomer10}, since they were selected to have largely orthogonal expression profiles. In other words, they were chosen since they were known to correspond to distinct and potentially interesting regions of the {\it{Platynereis}} brain. However, in many other settings a larger number of genes, many with correlated expression profiles (i.e., genes in the same regulatory network) will be profiled and this assumption will be invalid. Consequently, extending the model to allow for dependence structure in the emission distributions will be a critical challenge.
		\subsection{Shortcomings of the method}
		talk about the fact that hypothesis need to be validated. Also K is not necessarly the true number of cluster depending on the number of genes considered. Also quantitative data with poisson distibution to open on the conclusions
	



%*****************************************
%*****************************************
%*****************************************
%*****************************************
%*****************************************
