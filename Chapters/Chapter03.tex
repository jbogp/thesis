%************************************************
\chapter{Clustering tissues from single cell expression data and visualizing them in a 3D space}\label{ch:non_spatial_clustering_visualization} 
%************************************************
\section{Elements of clustering for biological tissues}
	\subsection{Why cluster?}
    - Single cell data has a lot of potential but:
    - Big data general problem
    - need to be able to come back from the single cell level to the tissue for : a consistency check and improve the knowleddge at the tissue level from single cell level

	\subsection{General considerations about clustering}
    - No perfect method
    - directed vs undirected
    - choosing the number of cluster, a key issue

\section{Visualizing clustering results in 3D with bioWeb3D}
	\subsection{Background}

Visualisation is a key feature in the analysis of large biological datasets, especially when analysing organized structures with distinct sub-clusters \cite{Rubel10}. This is particularly important when analysing 3-Dimensional (3D) datasets. When a biological process or feature has been described spatially by a set of 3D referenced points, either via laboratory work (confocal microscopy for example) or generated within a simulation, with some data attached to each point in space, the first step in interpreting the data is to visualise it. Once the data are visualised and the quality assessed, downstream analysis can proceed. For example, on obvious second step is to cluster the observations into different classes based upon the information assiciated with each point; those results will also need visualisation. \\

While various 3D visualisation tools have been developed, they have typically been made available via a locally installed piece of software such as BioLayout Express$^{3D}$\cite{Freeman07}, Arena3D \cite{Pavlopoulos08},  3D Genome Tuner \cite{Wang09}, the Allen Brain Atlas \cite{Lein07} or Cytoscape \cite{Shannon03}. Other 3D visualisation tools have been built online and are accessible through the browser directly, such as AstexViewer \cite{Hartshorn02}, which is utilised by the Protein Databank Europe via a Java Applet. More recently, visualisation tools developed using HTML5/WebGL capabilities have been described, although they have focused on very specific applications, such as analysing radiology data  \cite{Dinesh12}.\\
Importantly, as yet no tool has allowed biologists to view their own 3D data directly online in an easy, fast and interactive and secure way. Using webGL and the JavaScript 3D library Three.js, bioWeb3D aims to be a simple, generic, tool for tackling this problem.\\


	\subsection{Implementation}

bioWeb3D allows the user to represent any 3D dataset on their browser by defining only two files. The two files can either be formatted as JSON files, a widely used structured format on the web \cite{Wilde07} or directly as Comma Separated Values files (CSV).\\

The first file used by the application, referred to as the ``dataset file", contains the coordinates of every point in the dataset. The second type of file used, the ``information layer" file, describes one or several information layers that are associated with the points defined in the first file. For example, if each point defines the location of a cell within a tissue, the second file could describe whether a particular gene is expressed in each cell. That way the tissue expression profile can be represented in the spatial context of the tissue.\\

Datasets can be viewed and compared in up to four ``worlds" (each world refers to a separate visualisation sub-window) at the same time. Although browser based, the application, fully written in Javascript, does not need to send any data to the host server. Instead the modern internet browser's local file system reading capabilities are used through the HTML 5 FileReader functionality. This allows the application to handle, in a very short period of time, large datasets while ensuring that the privacy of the data is maintained.\\

Although the focus is on making bioWeb3D simple and easy to use, some options are available to customise how datasets are represented. The application can be used to visualise sequential information, such as 3D protein structures, in which case links can be drawn between the points. In other situations, such as when a population of cells is considered, the points can be left unlinked as individual particles. The information layers are visualised by colouring the 3D points according to the class that each point belongs to.

		\paragraph{technological overview}
bioWeb3D is fully written in HTML/Javascript. It relies heavily upon a relatively recent 3D javascript library called Three.js \cite{three}. This library is used as the main interface between webGL (cross-platform, royalty-free web standard for a low-level 3D graphics API) \cite{webgl}. More specifically, bioWeb3D allows the generation and manipulation of simple Three.js objects. Indeed the primary challenge associated with the creation of bioWeb3D has been to design interactions between the 3D visualisation and the user interface in the most efficient way.\\
		\paragraph{Defining the input files formats}
Using the JSON format to input files into bioWeb3D is recommended because of its rigorous structure, which allows fast Javascript object generation within the browser interpreter. Compared to other data-interchange languages, such as XML, JSON is also easily human readable thanks to a light-weight syntax. It is also supported by all of the primary internet browsers.\\
However, much data generated in the biological sciences is stored within CSV files. Converting CSV documents to the JSON format used in this application is not always trivial. In order to facilitate this process, the application is also able to interpret simple CSV files following a certain format as an input.


		\subsubsection{Dataset files specification}
When the user adds a new {\it{Dataset}} file, a new Dataset section is created in the ``Data" panel of the application. One raw data file contains one dataset.\\
		\paragraph{JSON format}
The {\it{dataset}} file should have a root object called ``dataset" which contains: \begin{itemize}
\item{The ``name" property of the dataset (\textit{e.g.}, ``my dataset");}
\item{The ``chain" parameter, which should be set to \textit{true} if the points are linked (the default value is \textit{false}) - the data will be considered sequentially, with each point linked according to its order in the dataset file;}
\item{The ``points" property, which is a two dimensional array representing a list of (x,y,z) vectors that define the co-ordinates of the points.}
\end{itemize}

Listing \ref{jsondataset} is an example of a minimal 3 points dataset file:

		\paragraph{CSV format}
Each line represents a point and the three coordinates on each line must be separated by ``comma" characters.\\As an example, listing \ref{csvdataset} carries the same information as the JSON file in Listing \ref{jsondataset}. Do note that although the spatial information remains the same it is not possible to set a name or to link the points within a CSV file input.



		\subsubsection{Information layer file specification}The {\it{Information layer}} file contains information about the points described in the Dataset file. The information in this file has to be given in the same order as the points defined in the Dataset file.

		\paragraph{JSON format}
The {\it{information layer}} files must have a root element named  ``information". Since one information file can define multiple information sets, the structure below ``information" is a list. Each element of the list is structured as follows:
\begin{itemize}
\item{The ``name" property (optional);}
\item{The ``numClass" property, which indicates the number of different classes the data will be assigned to;}
\item{The ``labels" property, which defines a list of names for the ``numClass" classes previously defined (optional);}
\item{The ``values" property, which defines the class of each point in the dataset. As points do not have single IDs, this property must be in the same order and have the same length as the points defined in the {\it{dataset}} file.}
\end{itemize}

For example coming back to the 3 points defined in Listing \ref{csvdataset} and Listing \ref{jsondataset}, two information layers could correspond to: 
\begin{itemize}

\item{one clustering algorithm that puts the first two points together in class one and the third point alone in a second class}
\item{a second clustering algorithm that puts each point in a separate class}
\end{itemize}

In this case the Information layer file would look like Listing \ref{jsoninfo}.


		\paragraph{CSV format}
Each column will represent in which class each point belongs. The separation character between columns must be a ``comma". Listing \ref{csvinfo} carries the same information as Listing \ref{jsoninfo}. Note that it is not possible to use the ``labels" or ``name" properties available in Listing \ref{jsoninfo} within a CSV information layer file. 




	\subsection{Results and Discussion}


		\paragraph{Basic usage}
	In figure 1 is visualised as an example, the brain the marine annelid {\it{Platynereis dumerilii}} each point represents a cell in the brain, the colour of which indicates the class it belongs to. The user can interact with the visualisation via an interface on the right of the screen, which contains three panels. In the ``dataset" panel, the user can choose the {\it{datasets}} and {\it{information layer}} files that should be represented in each world. This panel also allows the user to show/hide specific classes of the selected information layers. Each dataset file entered will create a new sub-panel where the user can input {\it{information layer}} files for that world. Selecting an {\it{information layer}} in the drop-down list will display the data in the current world and generate a list of classes that the user can modify regarding their visibility and colour. The ``View" panel enables the user to choose which of the worlds are shown on the screen, ranging from 1 to 4 simultaneous worlds. Finally, the ``Settings" panel provides the user with a number of options that affect all worlds and all datasets, such as modifying the axes scales.

		\paragraph{bioWeb3D and local software}
Many 3D visualisation software tools, most of which require local installation, exist and provide similar functionalities with standard 3D format input such as Wavefront .OBJ. Some are extremely generic and powerful like Blender. However, these tools are not typically oriented towards a scientific audience. Moreover, those that are more focused on science are often targeted towards a very specific application, especially in medical sciences \cite{Wang09}. In this context, we believe that bioWeb3D can be useful as it is completely generic and browser based. It should also be noted that recent browser improvements regarding GPU acceleration through the webGL paradigm allow bioWeb3D to visualise several hundred thousand points. Additionally, local software is usually platform specific, which is not the case for browser based applications.

		\paragraph{bioWeb3D and Java Applets}
As mentioned previously, browser based 3D visualisation tools currently exist mainly in the form of Java Applets. This technology has attracted much criticism in 2012 regarding security flaws, leading the ``United States Computer Emergency Readiness Team" to advise that all Java Applets should be disabled due to current and future Java vulnerabilities \cite{security}. The development of WebGL technology is viewed by many as a candidate for replacing Applets. 



		\paragraph{Current limitations}
The main current limitation of a webGL based application is the machine and browser compatibility. Only computers with fairly recent graphic cards will be able to run a 3D environment. It should also be noted that Microsoft has notified the developer community that Internet Explorer is not scheduled to support WebGL in the near future. However, importantly, Chrome, Firefox, Safari and Opera all now support webGL applications. It could also be important to mention for eventual future developments that webGL is supported on mobile platforms such as iOS or Android. \cite{caniuse}

	\subsection{Conclusions}
bioWeb3D is designed to be a simple and quick way to view 3D data with a specific focus on biological applications.  Being browser-based, the software can be easily used from any computer without the need to install a piece of software. Importantly bioWeb3D has been designed to offer a very straightforward and easy-to-use working environment. Despite the current limitations in terms of compatibility or rendering performances for large numbers of points, we believe that bioWeb3D will enable non-experts in 3D data representation to quickly visualise their data and the information attached to it in many biological context, thus facilitating downstream analyses.

	\subsection{Availability and requirements}
The full source code is available on the github page of the project \cite{github}. A live version of the software is online \cite{bioWeb3D}. You will require a graphical card and a browser with webGL capabilities to run bioWeb3D.

\section{Non spatial clustering methods}
	\subsection{Hierarchical clustering}
    - General description of the method
    - Computing distance matrix on binary expression data
    - Hclust method to use (full or partial)
    - Discuss the choosing the K with hClust (dendrogram is not informative)
 
	\subsection{Independent mixture models}
    - General statistical framework
    - Present the model
    - Gene independence hypothesis (this will be discussed further in the next chapter)
    - Likelihood of the model
    - EM algorithm to maximize the parameters (theta)
    - Choosing K with the BIC

\section{Discussion}
	\subsection{Spatial clustering techniques (hierarchical, model based)}
    - Limits of non spatial methods on noisy data (cite the "cell model of chapter 1" paragraph)
    - Not using the spatial data seems silly when we do have it (playing chess blind comparison ?)
    - Some clustering methods are able to take into account the spatial localization of the data points as well as the expression data
    - Quickly present spatial methods (spatial hclust), Mixture with a spatial component

	\subsection{Method chosen}
    - MRF because theta parameters are informative, easy to compute a likelihood and to choose K




%*****************************************
%*****************************************
%*****************************************
%*****************************************
%*****************************************
