%************************************************
\chapter{Capturing gene expression in Platynereis dumerilii's brain}\label{ch:background}
%************************************************
\section{Platynereis dumerilii, an ideal organism of brain development studies}
     \subsection*{General description}
    - Marine annelid
    - Place in evolution 
    - Link between fast evolving models (drosophila) and vertebrates
    - Phases of development : larvae and adult
    - Larvae and adult behavior

 
     \subsection*{development of the larvae}
    - Producing Platynereis in the lab
    - Stereotypical development of the lavae brain 
    - FIG 1 : two slices of the same region at 48hpf in 2 different individuals

\section{Gene expression in Platynereis' developing brain}
     \subsection*{Mechanisms of gene expression}
    - Generalities about gene expression
    - Genes, transcription factors, all cells have the same genome expression differs

     \subsection*{Platynereis' brain development}
    - General development pattern (symmetries, eyes, mushroom)
    - LOOK for a paper summing that up
    - State of the brain at 48hpf because we'll use it later

     \subsection*{Spatial organization of complex biological tissues like the brain}
    - Some tissues will be spatially well defined others will be scattered
    - Example with the pancreas ?
    - Introducing the idea of the spacial coherency heterogeneity

\section{Capturing gene expression in the laboratory}
     \subsection*{In-situ hybridization assays}
    - FIG 2 : In situ hybridization principles
    - Explanations about the technique 

     \subsection*{Building a referenced library of gene expression for Platynereis}
         - Stereotypical development allows one gene to be considered as reference
    - Different individuals are "replicates"
    - Mapping to a scaffold created by the reference gene

     \subsection*{RNA sequencing}
    - FIG 3 : about RNA sequencing for tissues
    - Explanations about the technique
    - Obtaining the full transcriptome at once
    - Necessity of having the genome to map to or a list of known genes (primR) 
    - Discuss the starting RNA quantity 
    - Discuss the fact that gene expression is averaged over the tissue losing spatial information.

%*****************************************
%*****************************************
%*****************************************
%*****************************************
%*****************************************




