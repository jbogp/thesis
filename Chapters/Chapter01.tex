%
\chapter{Capturing gene expression in \platyfull{}'s brain}\label{ch:background}
%
\section{Platynereis dumerilii, an ideal organism of brain development studies}
     \subsection{General description}
     \platy{} is a marine annelid of the class Polychaeta, it has been established as one of the main marine animal models in the fields of evolutionary, developmental and neurobiological biology as well as ecology and toxicology \cite{hutchinson95,tessmar03,hardege99,dorresteijn90,fischer04,Fischer10}. As a member of the bilateria \platy{} has a defined bilateral symmetry.\\
     
     \platy populates shallow (no more than 3m) hard ocean floors around the world. It is commonly found in the Mediterranean sea, the north Atlantic coast of Europe as well as in the shallow seas surrounding Sri Lanka, Java and the Philippines. Eggs, embryos and larvae are roughly 160 $\mu$m while the adults can measure up to 6cm in length.
     
\begin{figure}[bth]
        \myfloatalign
        \subfloat[Larval form of \platy{}. Image: MPI for Developmental Biology.]
        {\label{fig:platynereis_larvae}
        \includegraphics[width=.45\linewidth]{gfx/chapter1/platynereis_larva.jpg}} \quad
        \subfloat[Adult \platy{}. Image: Arendt group, EMBL]
        {\label{fig:platynereis_adult}%
         \includegraphics[width=.45\linewidth]{gfx/chapter1/platynereis_adult.jpg}}
        \caption{\platyfull{}'s larva and adult forms.}\label{fig:platynereis}
\end{figure}
     
     They are several reasons why \platy{} has been chosen as a model by numerous laboratories. In terms of evolution \platy{} shows several interesting characteristics. It belongs to the lophotrochozoan taxon of the bilaterian animals as opposed to most of the well established models animal which either belong to the ecdysozoans (\species{Caenorhabditis elegans}, \species{Drosophila melanogaster}) or the deuterostomes (mouse, human). Lophotrochozoans being extremely under represented, \platy{} as a model organism is essential to comparative approach on bilaterian biology.\\
     
     \platy{} also shows an exceptionally slow evolutionary lineage. It has even been described as a ``living fossil" for that reason \cite{Fischer10}. This means that the ancestral developmental characteristics of \platy{} are at an image of the common past of all bilaterians. To illustrate this fact an interesting example described in \cite{denes07,tessmar07} is the conserved molecular topography of the genes responsible for the development of the central nervous system between \platy{} and all vertebrates. This slow evolutionary rate confers \platy{} the advantage of being a link between fast evolving models like \species{drosophila} and vertebrates.\\
     
     In terms of practicality, \platy{} can easily be kept and bred in captivity producing offspring throughout the year \cite{fischer04}. The behavioural characteristics of \platy{} mating ritual have been well studied. The ``nuptial dance" happens on the water surface,  male and female releasing the sperm and eggs synchronously, respectively. This activity is synchronized by pheromones released into the water \cite{zeeck98}. Over 2000 individuals can be produced within a single batch. Every new individual will undergo embryonic then larval development before reaching \platy{}'s adult form.\\

 
     \subsection{Larval development}
    Similarly to the other polychaetes, the larval development of \platy{} can be decomposed into three main anatomical stages: the trochophore, the metotrochophore and the nectochaete. The trochophore is spherical and moves via a equatorial belt of ciliated cells as well as an apical organ possessing a ciliary tuft as seen on figure \ref{fig:platynereis_larvae} \cite{rouse99,nielsen04}. the metotrochophore stage is characterized by the development of a slightly elongated segmented trunk compared to that of the trochophore \cite{hacker98}. The next stage is the nectochaete larvae that resembles the adult (figure \ref{fig:platynereis_adult}) in most of the traits especially with parapodial appendages used for swimming and crawling \cite{hacker98}. This traditional subdivision has been applied to \platy{} \cite{hauenschild69}.\\
    
    Aside from this purely anatomical subdivision, an additional staging systems exists and has become the norm for current studies. The development is measured in \textit{hours post fertilization} (hpf) at 18^{\circ}C.
    
    A key factor making \platy{} such an interesting model to work with is the fact that after fertilization, the $\approx 2000$ larva will start developing at the exact same time, in a synchronous fashion. Furthermore, the larval development of \platy{} follows a very stereotypical pattern with very little variation from one individual to the other and even between batches provided the temperature is kept constant \cite{fischer04,dorresteijn90}. An example showing the similarity between individuals during development can be seen on figure \ref{fig:brain_comparison}. this is a very important feature as it allows biologists to repeat experiments on several individuals at a very close developmental stage even if they are from different batches.\\
    
\begin{figure}[bth]
  \includegraphics[width=\linewidth]{gfx/chapter1/brain_comparison.png}}
  \caption{\platyfull{}'s stereotypical and synchronous development. In green and red are two different \platy{} individuals' with the same gene expression being highlighted. They show extremely similar patterns of development.}
  \label{fig:brain_comparison}
\end{figure}

\section{Gene expression in Platynereis' developing brain}
     \subsection{Mechanisms of gene expression}
    - Generalities about gene expression
    - Genes, transcription factors, all cells have the same genome expression differs

     \subsection{Platynereis' brain development}
    - General development pattern (symmetries, eyes, mushroom)
    - LOOK for a paper summing that up
    - State of the brain at 48hpf because we'll use it later

     \subsection{Spatial organization of complex biological tissues like the brain}
    - Some tissues will be spatially well defined others will be scattered
    - Example with the pancreas ?
    - Introducing the idea of the spacial coherency heterogeneity

\section{Capturing gene expression in the laboratory}
     \subsection{In-situ hybridization assays}
    - FIG 2 : In situ hybridization principles
    - Explanations about the technique 

     \subsection{Building a referenced library of gene expression for Platynereis}
         - Stereotypical development allows one gene to be considered as reference
    - Different individuals are "replicates"
    - Mapping to a scaffold created by the reference gene

     \subsection{RNA sequencing}
    - FIG 3 : about RNA sequencing for tissues
    - Explanations about the technique
    - Obtaining the full transcriptome at once
    - Necessity of having the genome to map to or a list of known genes (primR) 
    - Discuss the starting RNA quantity 
    - Discuss the fact that gene expression is averaged over the tissue losing spatial information.

%
%
%
%
%




