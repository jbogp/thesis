%%%%%%%%%%%%%%%%%%%%%%%%%%%%%%%%%%%%%%%%%
% Short Sectioned Assignment
% LaTeX Template
% Version 1.0 (5/5/12)
%
% This template has been downloaded from:
% http://www.LaTeXTemplates.com
%
% Original author:
% Frits Wenneker (http://www.howtotex.com)
%
% License:
% CC BY-NC-SA 3.0 (http://creativecommons.org/licenses/by-nc-sa/3.0/)
%
%%%%%%%%%%%%%%%%%%%%%%%%%%%%%%%%%%%%%%%%%

%----------------------------------------------------------------------------------------
%	PACKAGES AND OTHER DOCUMENT CONFIGURATIONS
%----------------------------------------------------------------------------------------

\documentclass[paper=a4, fontsize=11pt]{scrartcl} % A4 paper and 11pt font size

\usepackage[T1]{fontenc} % Use 8-bit encoding that has 256 glyphs
\usepackage{fourier} % Use the Adobe Utopia font for the document - comment this line to return to the LaTeX default
\usepackage[english]{babel} % English language/hyphenation
\usepackage{amsmath,amsfonts,amsthm} % Math packages

\usepackage{lipsum} % Used for inserting dummy 'Lorem ipsum' text into the template

\usepackage{sectsty} % Allows customizing section commands
\usepackage[cm]{fullpage}

\allsectionsfont{\centering \normalfont\scshape} % Make all sections centered, the default font and small caps

\usepackage{fancyhdr} % Custom headers and footers
\pagestyle{fancyplain} % Makes all pages in the document conform to the custom headers and footers
\fancyhead{} % No page header - if you want one, create it in the same way as the footers below
\fancyfoot[L]{} % Empty left footer
\fancyfoot[C]{} % Empty center footer
\fancyfoot[R]{\thepage} % Page numbering for right footer
\renewcommand{\headrulewidth}{0pt} % Remove header underlines
\renewcommand{\footrulewidth}{0pt} % Remove footer underlines
\setlength{\headheight}{5	pt} % Customize the height of the header

\numberwithin{equation}{section} % Number equations within sections (i.e. 1.1, 1.2, 2.1, 2.2 instead of 1, 2, 3, 4)
\numberwithin{figure}{section} % Number figures within sections (i.e. 1.1, 1.2, 2.1, 2.2 instead of 1, 2, 3, 4)
\numberwithin{table}{section} % Number tables within sections (i.e. 1.1, 1.2, 2.1, 2.2 instead of 1, 2, 3, 4)

%\setlength\parindent{0pt} % Removes all indentation from paragraphs - comment this line for an assignment with lots of text

%----------------------------------------------------------------------------------------
%	TITLE SECTION
%----------------------------------------------------------------------------------------

\newcommand{\horrule}[1]{\rule{\linewidth}{#1}} % Create horizontal rule command with 1 argument of height

\title{	
\normalfont \normalsize 
\horrule{0.5pt} \\[0.4cm] % Thin top horizontal rule
\huge Spatial analysis of complex biological tissues from single cell gene expression data \\ % The assignment title
\horrule{2pt} \\[0.5cm] % Thick bottom horizontal rule
}

\author{Jean-Baptiste Olivier Georges Pettit} % Your name


\begin{document}

\maketitle % Print the title


The understanding of life in terms of biological function is a highly complex and crucially important subject for basic research and applied research through medicine and healthcare. The way scientists look at biological tissues has greatly evolved throughout time and has recently been revolutionized following the technological advances leading to the sequencing of the human genome. The \emph{-omics} fields are nowadays central to the understanding of biology. Perhaps the most tantalizing aspect of molecular biology is found in the field of transciptomics -the study of gene expression- as indeed this process defines the life of each cell during development and throughout the life of all organisms.\\

Complex multicellular organisms exhibit a large number of diverse cell types (several hundred in mammals). Those cell types are often defined with regard to their anatomical traits (i.e their shape, size and other visible characteristics) as well as their functional traits (endocrine, muscular, etc function). However, these traits are purely descriptive and somewhat ambiguous as convergence phenomenons can lead different cell types to exhibit the same properties. Therefore the study of gene expression patterns opens the way for unbiased methods to characterize cell types. This has become especially true since the scale of study in gene expression assays is shifting from the tissue level to the single cell level, enabling cell-to-cell heterogeneity to be described.\\

This thesis revolves around such single cell gene expression datasets in the context of the marine annelid or ragworm \emph{Platynereis dumerilii}, an important model organism, part of the lophotrochozoan taxon of the bilaterians. After describing single cell expression data acquired from Wholemount in Situ Hybridization assays for 169 genes as well as single cell RNA-seq data for 72 cells in the developing brain 48 hours post fertilization of \emph{P. dumerilii}, I discuss the main advantages of both methods and propose a back-mapping method to generate a spatially referenced data set of whole transcriptomes at the single cell level.\\

As the spatial characteristics of the data are crucial to the work presented in this thesis, I also present a 3-dimensional visualization tool that facilitates greatly the upstream and downstream analysis of such datasets. The rest of the thesis focuses on answering the question of identifying cell types from single cell expression data, that is clustering cells together in a meaningful, functional way. After describing data-only driven existing methods to answer this question, I propose a statistical method based on Hidden Markov random fields to cluster the cells according to their gene expression patterns as well as their spatial localization in the brain. The method is validated by a simulation study and the quality of the results are compared to those of other clustering methods. Finally the method's results when applied to the \emph{P. dumerilii} in-situ hybridization dataset are biologically validated and functional hypothesis about putative unstudied regions of the brain and their function are formulated.






%----------------------------------------------------------------------------------------

\end{document}




